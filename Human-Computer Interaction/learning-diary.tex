\documentclass{article}
\title{HCI Learning Diary}
\author{Candidate Number: 18512}

\begin{document}
\maketitle
\section*{Entry 1}

\noindent I was initially quite sceptical about the HCI module, I tend to prefer more mathematical subjects, where concepts are objective rather than subjective. I have typically done better in those kinds of modules. To be honest the first week of HCI has not put me at ease. 'Good' and 'Bad' design are concepts that are very difficult to pin down and I worry that if I am required to identify such things in an exam I will not do well.
\\\indent I thought it was interesting to see the lecture slides with pictures of old computer systems, like the punch-card computer at NASA and the old Apple Mac computer. I am interested to hear the lecturer's thoughts on punch card systems. I would also like to know if she has any ideas about programming languages in relation to HCI. Perhaps I will send an email.

\section*{Entry 2}

\noindent Sending an email to Kate Howland has been helped with my concerns about the module. I now seems that the subject is more objective than I realised.
\\\indent It is interesting to see just how relative our judgements about effective design. As Kate said in her email, we have no way of judging designs in a vacuum. When the punch card system was developed, or technological ability limited the degree to which the design could be user-centered.
\\\indent The Tivo remote discussed in the text-book is an interesting example. There is certainly a problem with most TV remotes. I think the minimal approach that the Tivo remote takes could be taken even further - I think a 4-way navigation button, a select button, a power button and volume buttons would be ideal. This bare minimum allows all other functionality to be managed as part of the on-screen GUI. This would eliminate the problem that reading-glasses wearers have - they require glasses to see the remote buttons but not the TV screen.
\\\indent The Tivo remote reminds me of an example of an interface with awful feedback - my freesat box at home takes far too long to respond to button presses on the remote, sometimes as much as 10 seconds. It makes you think that the press was not detected, so you press it again, and then all of a sudden, it registers both presses. If you're scrolling through the TV guide and you were trying to go down a page, this means you accidentally go down two pages, which is very frustrating.

\section*{Entry 3}

\noindent For my group project this week, my 'homework' was to review an existing study assistant app, with a focus on usability goals. My chosen app is 'Papyrus', a handwriting/note-taking app for android.
\\\indent \textbf{Effectiveness} - 4 stars. The pinch-to-pan/zoom feature allows you to edit writing either in fine detail, or in a less considered way.
\\\indent \textbf{Efficiency} - 1 star. While the zooming/panning is very effective, it does take a long time to actually input writing, because there is so much panning and zooming required to manipulate the page in order to be ready to enter text. Often when trying to pan \& zoom, only one finger is registered, and instead of panning or zooming, a line is drawn. You then have to press 'undo', and try again. This is frustrating and slows things down considerably.
\\\indent \textbf{Learnability} - 4 stars. Menus are very short, settings are well explained and not too extensive. The brush size/colour options are familiar and recalling how to use them is trivial.
\\\indent \textbf{Memorablity} - 4 stars. The points from learnability apply here. There is so little to learn about the app that memorising the steps to achieve tasks is trivial.
\\\indent \textbf{Safety} - 4 stars. There is a delete note/notebook confirmation which prevents accidental deletion. There are undo and redo buttons in the editor to undo mistakes. One potential (but rather unlikely) issue would be to make a mistake, and then leave the note. When you re-enter the note, you cannot undo previous changes. If this option were added, this would not be an issue.
\\\indent \textbf{Utility} - 3 stars. A good range of options relating to handwriting entry, such as brush size, colour etc. Ability to sort notes into different categories, however one note cannot be in multiple categories. The app lacks anything beyond handwriting entry - on-screen keyboard entry would be very useful due to the slowness that results from the need to pan \& zoom.
\\\indent \textbf{Summary} - Due to the inefficiency of the app's primary task, the app is effectively useless. The idea of the app is that handwriting entry is faster than using an on-screen keyboard, and therefore is a good replacement for it when speed is an issue, is completely undermined by the slowness that results from panning and zooming so often.

\section*{Entry 10}

\noindent It is interesting to consider Mark Weiser's vision of the future of computing. We have certainly not yet come close to a realisation of this vision - However I would say that smartphones \& tablets represent progress towards his idea of 'pads'. In his seminal paper, 'The Computer for the 21st Century', his description of the functionality of pads is not too far from modern day tablets. However, in contrast to today's tablets, he describes pads as 'scrap computers, analogous to scrap paper' - this could not be further from today's reality. Tablets tend to be highly personal devices, and are prized highly by their owners, and could almost be described as fashion accessories. Despite this, it is not difficult to imagine that as the cost of these devices drops, this will change, so that we might be able to 'Spread many electronic pads around on the desk, just as you spread out papers' as Weiser describes.
\\\indent I found the discussion of the appropriateness of the design principles for ubiquitous computing interesting. The image of the 'Flow Blocks' stood out to me in particular as an example of where the design principles break down.
\\\indent With objects like these flow blocks, the concept of mapping becomes less well defined. Mapping dictates that there should be a clear logical connection between controls and real world results. With the flow blocks, the control mechanism and the real world results are one and the same thing, so essentially we get mapping 'for free' with devices like this.
\\\indent Constraints also 'come for free' with the flow blocks - it is clear exactly what you cannot do with the blocks simply by their 3D shape - if you cannot fit two blocks together in a certain way, then they are not intended to be connected that way. It seems a similar rule applies for both visibility and feedback also applies. Perhaps this is because the design principles lead us to designs that mimic real world objects like chairs and paper, and so when our interfaces \emph{are} real world objects, they abide by those principles inherently. So perhaps we do need new design principles, not because the existing ones are invalid, but because they are inherent in 3D, real world interfaces, and we should instead focus our attention on other things.

\end{document}
